\documentclass{article}

\newcommand{\matiere}{Math} %The name of the subject
\newcommand{\titre}{Assignment 5} %The name of the assignment
\newcommand{\auteur}{Your Name}
\newcommand{\ladate}{}%If you wan to specify the date
\newcommand{\questionformat}{1} %Subquestions numbering : A,I,1
\newcommand{\subquestionformat}{i} %Subquestions numbering : A,a,I,i,1

%Optional
%\newcommand{\questionname}{Problem} %If you want to rename "Question"

\usepackage{./template}

\begin{document}
\maketitle

\question

\subquestion

\begin{eq}This is an equation\end{eq}
\begin{eq*}This equation is not numbered\end{eq*}

\begin{res}This is a result\end{res}
\begin{res}This result is not numbered\end{res}


\subquestion

\begin{prop}This is a property\end{prop}
\begin{prop*}This property is not numbered\end{prop*}

\begin{notation}This is a notation\end{notation}
\begin{notation*}This notation is not numbered\end{notation*}

\begin{hypo}This is an hypothesis\end{hypo}
\begin{hypo*}This hypothesis is not numbered\end{hypo*}

\begin{conclusion}This is a conclusion\end{conclusion}
\begin{conclusion*}This conclusion is not numbered\end{conclusion*}


\subquestion[1]

The numbering of this subquestion is different from the rest.

We can call an earlier block using \ref{Equation:1.1}.

\begin{eq}[A great equation]This equation has a title\end{eq}

\question


\subquestion

Foo

\subquestionnobar

A sub-question without limiter


\questionnobar

A question without limiter.

\pulse{5,555e-2}

\metre{17}

\temps{0.1e-2}

\invfr{a}

\ddrond{g}{x}

$\int{a}{b}f(z)dz$

\do

\inv{ab}

\vec{u}\dpi

\deriv{f}{z}
\drond{f}{x}

\c{\frac{\pi z}{L}}

\s{\frac{\pi z}{L}}

\ux

\ex

\sqfr{3}{2} \ei{\omega t}\w

\end{document}

